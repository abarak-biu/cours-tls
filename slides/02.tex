% Written in 2015 by Manuel Pégourié-Gonnard
% SPDX-License-Identifier: CC-BY-SA-4.0

\documentclass{mpg-ep-slides}

\author[MPG]{Manuel Pégourié-Gonnard\\
  \href{mailto:mpg@elzevir.fr}{{mpg@elzevir.fr}}
}
\institute[ARM]{\normalsize ARM France - IoT - mbed TLS}
\license{%
  \url{https://github.com/mpg/cours-tls} \\[.5ex]
  \href{https://creativecommons.org/licenses/by-sa/4.0/}{CC-BY-SA 4.0}
}

\title{Cours de cryptologie appliquée de l'EPITA \\ TLS - partie 2}
\date{23 novembre 2015}

\DeclareMathOperator\RSA{RSA}
\DeclareMathOperator\RSAPKCS{RSA\_PKCS1\_15}
\DeclareMathOperator\PKCS{PKCS1\_15}

\begin{document}

\lictitle

\section[Handshake]{Poignée de mains : généralités}
\tocsect

\begin{frame}{Rappel}
  % cf frame "Les couches de TLS" partie 1
  \begin{center}
    \renewcommand\arraystretch{1.5}
    \begin{tabular}{|c|c|c|c|}
      \hline
      Handshake & ChangeCipherSpec & Alert &
      \multicolumn{1}{|c|}{ \color{gray} HTTP, SMTP, etc. } \\ \hline
      \multicolumn{4}{|c|}{Record} \\ \hline
      \multicolumn{4}{|c|}{\color{gray} TCP, UDP} \\ \hline
    \end{tabular}
  \end{center}

  \begin{block}{Handshake}
    \begin{itemize}
      \item Négociation \emph{fiable} des paramètres
      \item Établissement de clé de session \emph{secrètes}
      \item \emph{Authentification} du serveur, voire du client
      \item Crypto (en général) asymétrique + symétrique
    \end{itemize}
  \end{block}
  \begin{block}{Record}
    \begin{itemize}
      \item \emph{Confidentialité} et \emph{intégrité} des données
      \item Crypto symmétrique uniquement
    \end{itemize}
  \end{block}
\end{frame}

\begin{frame}[containsverbatim]{Flot général}
  \begin{Verbatim}[gobble=4, fontsize=\small]
    ClientHello          -------->
                                            ServerHello
                                           Certificate*
                                     ServerKeyExchange*
                                    CertificateRequest*
                         <--------      ServerHelloDone
    Certificate*
    ClientKeyExchange
    CertificateVerify*
    [ChangeCipherSpec]
    Finished#            -------->
                                     [ChangeCipherSpec]
                         <--------            Finished#
    Application Data#    <------->    Application Data#
  \end{Verbatim}
  \texttt{*} = optionel, \texttt{\#} = chiffré
\end{frame}

\begin{frame}[containsverbatim]{Encodage des messages}
  \begin{center}
    \renewcommand\arraystretch{1.2}
    \begin{tabular}{|c|c|c|c|p{8em}|}
      \hline
      T & L1 & L2 & L3 & \centering \emph{(contenu)} \cr
      \hline
    \end{tabular}
  \end{center}

  \begin{Verbatim}[gobble=4, fontsize=\footnotesize]
    struct {
        HandshakeType msg_type;    /* handshake type */
        uint24 length;             /* bytes in message */
        select (HandshakeType) {
            case client_hello:        ClientHello;
            case server_hello:        ServerHello;
            [...]
        } body;
    } Handshake;

    enum {
        client_hello(1), server_hello(2), [...]
        finished(20), (255)
    } HandshakeType;
  \end{Verbatim}
\end{frame}

\begin{frame}[containsverbatim]{ClientHello}
  \begin{Verbatim}[gobble=4, fontsize=\scriptsize]
    struct {
        ProtocolVersion client_version;
        Random random;
        SessionID session_id;
        CipherSuite cipher_suites<2..2^16-2>;
        CompressionMethod compression_methods<1..2^8-1>;
        select (extensions_present) {
            case false:
                struct {};
            case true:
                Extension extensions<0..2^16-1>;
        };
    } ClientHello;

    struct {
        uint32 gmt_unix_time;
        opaque random_bytes[28];
    } Random;
    opaque SessionID<0..32>;
    uint8 CipherSuite[2];
    enum { null(0), (255) } CompressionMethod;
  \end{Verbatim}
\end{frame}

\begin{frame}[containsverbatim]{ServerHello}
  \begin{Verbatim}[gobble=4, fontsize=\scriptsize]
    struct {
        ProtocolVersion server_version;
        Random random;
        SessionID session_id;
        CipherSuite cipher_suite;
        CompressionMethod compression_method;
        select (extensions_present) {
            case false:
                struct {};
            case true:
                Extension extensions<0..2^16-1>;
        };
    } ServerHello;

    struct {
        ExtensionType extension_type;
        opaque extension_data<0..2^16-1>;
    } Extension;
    enum {
        signature_algorithms(13), (65535)
    } ExtensionType;
  \end{Verbatim}
\end{frame}

\begin{frame}{Remarques sur les Hello}
  \begin{block}{Négociation}
    \begin{itemize}
      \item Le client propose, le serveur dispose
      \item Le serveur DOIT ignorer silencieusement les valeurs inconnues
        (version, suites, compression, extensions)
    \end{itemize}
  \end{block}

  \begin{block}{Suites}
    \begin{itemize}
      \item TLS_\emph{KEYEX}_WITH_\emph{CIPHER}_\emph{HASH}
      \item Ex recommandé : TLS_ECDHE_RSA_WITH_AES_128_GCM_SHA256
      \item OpenSSL a des noms différents, ex AES128-SHA
      \item Plus de 300 définies, seulement une poignée recommandée
      \item \url{https://www.iana.org/assignments/tls-parameters/tls-parameters.xhtml\#tls-parameters-4}
    \end{itemize}
  \end{block}
\end{frame}

\begin{frame}[containsverbatim]{Dérivation des clés}
  \begin{itemize}
    \item L'échange de clé fourni un secret pré-maître (PMS) de longueur
      variable
    \item On en dérive un secret maître (MS) de 48 octets :
      \begin{Verbatim}[gobble=8]
        MS = PRF(PMS, "master secret",
                 ClientHello.random + ServerHello.random)
                 [0..47];
      \end{Verbatim}
    \item La fonction pseudo-aléatoire (PRF) est basée sur des hash (MD5+SHA1,
      SHA-256) et est utilisée comme KDF
    \item Un bloc de clés est dérivé du MS en appliquant la PRF
    \item Il est découpé en clés et IV (client/serveur,
      AES/HMAC)
    \item RFC 7627 et TLS 1.3 remplacent les random par \emph{session hash}
      (hash de toute la poignée de main) (cf \emph{triple handshake})
  \end{itemize}
\end{frame}

\begin{frame}[containsverbatim]{Le message Finished}
  \begin{Verbatim}[gobble=4, fontsize=\footnotesize]
    struct {
        opaque verify_data[verify_data_length];
    } Finished;

    verify_data
       PRF(master_secret, finished_label, Hash(handshake_messages))
          [0..verify_data_length-1];
  \end{Verbatim}

  \begin{itemize}
    \item En pratique \texttt{verify_data_length} est 12 octets
    \item Le but est d'assurer la fiabilité de la négociation : un attaquant
      qui essaie d'influencer la négociation sera détecté
    \item Sert aussi de confirmation de clé, ce qui simplifie les preuves de
      sécurité
  \end{itemize}
\end{frame}

\begin{frame}{Serveur intolérants et attaque en downgrade}
  \begin{itemize}
    \item Des serveurs codés avec les pieds rejettent les ClientHello avec
      une version « trop » élevée, ou avec des extensions
    \item Certaines middlebox font de même
    \item Les clients qui veulent quand même communiquer dans ce cas font un
      \emph{fallback} avec un ClientHello.version plus faible
    \item Ce fallback hors-protocole n'est pas protégé par le Finished
    \item Un serveur intolérant est indistiguable d'un attaquant
    \item Ce fallback n'est donc \emph{pas sûr} (ex. partie DLE de POODLE)
    \item On a un compromis utilisabilité-sécurité (pour changer !)
    \item Le RFC 7507 améliore la situation avec une \emph{signaling
        ciphersuite value} (SCSV) indiquant le fallback
  \end{itemize}
\end{frame}

\section[RSA]{Transport de clé RSA}
\tocsect

\begin{frame}[containsverbatim]{Principe}
  \begin{Verbatim}[gobble=4, fontsize=\small]
    ClientHello          -------->
                                            ServerHello
                                            Certificate
                         <--------      ServerHelloDone
    ClientKeyExchange
    [ChangeCipherSpec] ...

      struct {
          ProtocolVersion client_version;
          opaque random[46];
      } PreMasterSecret;
      struct {
          public-key-encrypted PreMasterSecret pre_master_secret;
      } EncryptedPreMasterSecret;
      struct {
          EncryptedPreMasterSecret;
      } ClientKeyExchange;
  \end{Verbatim}
\end{frame}

\begin{frame}{L'attaque de Bleichenbacher (1)}
  \begin{block}{Oracle de padding sur PKCS\#1 v1.5}
    \begin{itemize}
      \item Rappel : $\RSAPKCS(m) =
        \RSA(\PKCS(m))$
      \item \(
          \PKCS(m) =
          \begin{array}{|c|c|c|c|c|} \hline
            \texttt{00} & \texttt{02} & \text{pad} & \texttt{00} & m \\ \hline
          \end{array}
        \)
      \item Autrefois, les serveurs envoyaient une erreur spécifique si le
        padding était incorrect
      \item L'attaquant envoie un chiffré $c'$ arbitraire, observe la réponse
      \item Si pas d'erreur de padding (une fois sur environ $2^{16}$), alors
        $m' = \RSA^{-1}(c')$ commence par \texttt{00 02}.
      \item Autrement dit, dans ce cas,
        $2 \cdot 2^{l-16} \le m' < 3 \cdot 2^{l-16}$
        où $l$ est la longueur en bits de la clé.
    \end{itemize}
  \end{block}

  \begin{block}{Propriété algébrique de RSA}
    RSA est un homomorphisme : $\RSA(m_1 m_2) = \RSA(m_1) \RSA(m_2)$
  \end{block}
\end{frame}

\begin{frame}{L'attaque de Bleichenbacher (2)}
  \begin{block}{Principe de l'attaque}
    \begin{itemize}
      \item On connaît $c$ et on va trouver $m$ en consultant l'oracle
      \item On pose $B = 2^{l-16}$, on sait déjà que $m \in [2B, 3B-1]$
      \item On cherche $s_1$ tel que $c \cdot s_1^e$ soit
        accepté par l'oracle ; on a
        \begin{align}
          m \cdot s_1 \mod n
          & \in [2B, 3B-1]
          \\ m \cdot s_1
          & \in \cup_{r \in \mathbb N} [2B + rn, 3B-1 + rn]
          \\ m
          & \in \cup_{r \in \mathbb N}
          \left[ \frac{2B + rn}{s_1}, \frac{3B-1 + rn}{s_1} \right]
          \cap [2B, 3B-1]
        \end{align}
      \item On itère (voir gribouillis au tableau)
      \item Au total quelques millions de requêtes suffisent : faisable
    \end{itemize}
  \end{block}
\end{frame}

\begin{frame}{L'attaque de Bleichenbacher (3)}
  \begin{block}{Comment traiter EncryptedPreMasterSecret}
    \begin{itemize}
      \item Naïf : \only<2->{\st}{si mauvais padding, erreur} ;
        \only<3->{\st}{si mauvaise version, erreur}
      \item<2-> Attaque de Bleichenbacher (1998) : oracle de padding
      \item<2-> TLS 1.0 : Si mauvais padding, on génère un PMS aléatoire,
        erreur reportée au Finished ; (\only<3->{\st}{si mauvaise version
          erreur})
      \item<3-> Attaque de Klima (2003) : oracle de version
      \item<3-> TLS 1.1 : \only<4->{\st}{Si mauvais padding, on génère un
          PMS aléatoire et on l'utilise} ; si mauvaise version pareil
      \item<4-> TLS 1.2 : On génère un PMS aléatoire ; si mauvais padding ou
        mauvaise version, ou l'utilise
      \item<4-> Usenix 2014 : new Bleichenbacher side channels
      \item<4-> Voir \texttt{ssl\_parse\_encrypted\_pms()} dans mbed TLS
      \item<5-> La « vraie » solution serait d'utiliser OAEP
    \end{itemize}
  \end{block}
\end{frame}

\begin{frame}{Concept de (Perfect) Forward Secrecy}
  \begin{block}{Problème avec le transport de clé RSA}
    \begin{itemize}
      \item Un attaquant enregistre plein de communications
      \item Plus tard, il obtient la clé RSA du serveur (faille sur le
        serveur, national security letter, factorisation, \dots)
      \item Il peut alors déchiffer toutes les communications enregistrées
    \end{itemize}
  \end{block}

  \begin{block}{Forward Secrecy}
    \begin{itemize}
      \item C'est quand l'attaque précédente n'est pas possible
        \texttt{\string:-)}
      \item Offerte par les suites basées sur Diffie-Hellman
    \end{itemize}
  \end{block}
\end{frame}

\section[(EC)DHE]{Échange de clé Diffie-Hellman (elliptique)}
\tocsect

\begin{frame}{Rappel : crypto sur les courbes elliptiques}
  \begin{columns}
    \column[t]{0.5\textwidth}
    \begin{block}{Log discret dans les corps finis (FFDL) \strut}
      \begin{itemize}
        \item $s = g^a \mod p$, trouver $a$
        \item Échange de clé : (FF)DH
        \item Signatures : DSA
        \item Complexité $\approx 2^{c \sqrt[3]{n}}$
      \end{itemize}
    \end{block}

    \begin{block}{Factorisation des entiers}
      \begin{itemize}
        \item $n = pq$, trouver $p$ et $q$
        \item Signature : RSA
        \item Chiffrement : RSA
        \item Complexité $\approx 2^{c \sqrt[3]{n}}$
      \end{itemize}
    \end{block}

    \column[t]{0.5\textwidth}
    \begin{block}{Log discret dans les courbes elliptiques (ECDL) \strut}
      \begin{itemize}
        \item $S = a \cdot G \in E$, trouver $a$
        \item Échange de clé : ECDH
        \item Signatures : ECDSA
        \item Complexité $\approx 2^{n / 2}$
      \end{itemize}
    \end{block}

    \begin{block}{Tailles équivalentes}
      \begin{tabular}{rrr}
        \toprule
        Sym. & FF/RSA & ECC \\
        \midrule
        112 &  2432 & 224 \\
        128 &  3248 & 256 \\
        256 & 15424 & 512 \\
        \bottomrule
      \end{tabular}
    \end{block}
  \end{columns}
\end{frame}

\begin{frame}[containsverbatim]{DHE-RSA, ECDHE-RSA, ECDHE-ECDSA}
  \begin{Verbatim}[gobble=4, fontsize=\small]
    ClientHello (+ext)   -------->
                                            ServerHello
                                            Certificate
                                      ServerKeyExchange
                         <--------      ServerHelloDone
    Certificate
    ClientKeyExchange
    [ChangeCipherSpec] [...]
  \end{Verbatim}
  \begin{itemize}
    \item ServerKeyExchange contient :
      \begin{itemize}
        \item les paramètres choisis ($p$ et $g$ pour FFDH, ou un identifiant
          de courbe elliptique)
        \item la part du serveur $g^a$ ou $aG$
        \item la signature du tout (plus les random) par le clé associée au
          certificat du serveur (authentification du serveur)
      \end{itemize}
    \item ClientKeyExchange contient la part $g^b$ ou $bG$ du client
  \end{itemize}
\end{frame}

\begin{frame}{Propriétés de (EC)DHE-(RSA|ECDSA)}
  \begin{itemize}
    \item Les trois échanges offrent la \emph{forward secrecy}
    \item Le « E » de (EC)DHE signifie éphémère : les parties génèrent un
      nouvel exposant secret à chaque fois
    \item DHE-RSA est environ trois fois plus coûteux que RSA côté serveur, et
      20 fois plus côté client
    \item ECDHE-RSA est seulement 15\% plus cher que RSA côté serveur (3 fois
      côté client)
    \item Chiffres de 2011, les courbes elliptiques progressent encore
    \item ECDSA est moins cher que RSA côté serveur, mais plus cher côté client
    \item ECDHE-(RSA|ECDSA) sont les plus recommandés
    \item ECHDE-RSA est actuellement le plus courant car les certificats ECDSA
      ne sont pas disponibles depuis longtemps
  \end{itemize}
\end{frame}

\begin{frame}{Problèmes avec FFDH (1)}
  \begin{block}{Validation des paramètres}
    \begin{itemize}
      \item Un serveur malicieux peut choisir des mauvais paramètres
      \item Le client ne peut pas les valider (trop coûteux)
      \item Utilisé dans une variante de l'attaque \emph{triple handshake}
      \item Solution : draft-ietf-tls-negotiated-ff-dhe
    \end{itemize}
  \end{block}
  \begin{block}{Taille des paramètres}
    \begin{itemize}
      \item Certains clients (Java 6) ne supportent pas plus de 1024 bits
      \item Le serveur ne peut pas savoir
      \item Compromis sécurité-interopérabilité
      \item Solution : draft-ietf-tls-negotiated-ff-dhe
    \end{itemize}
  \end{block}
\end{frame}

\begin{frame}{Problèmes avec FFDH (2)}
  \begin{block}{Réutilisation des paramètres}
    \begin{itemize}
      \item Après un gros précalcul pour un $p$ donné, casser des log discrets
        $\mod p$ devient très rapide (cf. \url{https://weakdh.org/} point 2)
      \item C'est probablement tout juste possible pour des gouvernements de
        casser quelques $p$ de 1024 bits.
      \item Beaucoup de serveurs ont un $p$ de 1024 bit en commun :
        \begin{itemize}
          \item Le plus courant est utilisé par 18\% des serveurs HTTPS
          \item Les deux plus courants donnent 60\% des VPN et 26\% des
            serveurs SSH
        \end{itemize}
      \item Solutions : utiliser un $p$ assez grand, ou unique
    \end{itemize}
  \end{block}
  Plus généralement sur les attaques groupées, voir :
  \url{http://blog.cr.yp.to/20151120-batchattacks.html}
\end{frame}

\section[PKIX]{Infrastructure à clé publique X.509}
\tocsect

\begin{frame}{Principe}
  \begin{itemize}
    \item Le serveur présente un certificat contenant une clé publique et une
      identité (et des méta-donnés)
    \item Il prouve qu'il possède la clé privée associée en déchiffrant le
      PMS ou en signant le ServerKeyEchange avec cette clé
    \item L'association entre l'identité et la clé est garantie par une
      autorité de certification (CA) qui a signé le certificat
    \item Cette CA peut elle-même avoir été signée par une autre CA, mais on
      ne fait que déplacer le problème
    \item Au bout d'un moment on doit arriver à une racine de confiance
    \item Il suffit de distribuer quelques racines de confiance avec les
      navigateurs pour authentifier des millions de sites web
  \end{itemize}
\end{frame}

\begin{frame}{Premiers problèmes}
  \begin{block}{Il faut vérifier les certificats (nom et signature)}
    \begin{itemize}
      \item Le nom dans le certificat doit correspondre au nom attendu
      \item Le certificat doit être signé correctement
      \item La liste des racines de confiance doit être censée et à jour
      \item Certaines implémentations ne vérifient rien par défaut
      \item Certains développeurs sont juste paresseux, ou se disent que c'est
        pas grave pour les tests, puis oublient en prod
    \end{itemize}
  \end{block}

  \begin{block}{Aparté : les trois états, dont un maudit, en sécurité}
    \begin{overprint}
      \begin{enumerate}
        \item Ça ne marche pas (et c'est pas sécurisé)
        \item \only<2->{\st}{Ça marche mais c'est pas sécurisé}
          \only<2->{JAMAIS! mieux vaut 1.}
        \item Ça marche et c'est sécurisé
      \end{enumerate}
    \end{overprint}
  \end{block}
\end{frame}

\begin{frame}{Le problème de la révocation}
  Quand une clé est compromise, le certificat associé doit être révoqué.

  \begin{block}{Les solutions qui marchent moyen}
    \begin{itemize}
      \item Certificate revocation list (CRL) : volume, latence
      \item Online Certificate Status Protocol : que faire si le serveur ne
        répond pas ? Attaque ou simple problème de disponibilité ?
      \item Les CAs n'ont pas d'incitation économique directe à investir
        beaucoup ici dans la révocation
    \end{itemize}
  \end{block}

  \begin{block}{De meilleures solution partiellement déployées}
    \begin{itemize}
      \item OCSP stapling : le serveur attache une réponse OCSP dans la
        poignée de main TLS
      \item CRLsets : les fabricants de navigateurs distribuent des diffs de
        CRL (Chrome)
    \end{itemize}
  \end{block}
\end{frame}

\begin{frame}{Les autorités « de confiance »}
  \begin{block}{Le maillon faible}
    \begin{itemize}
      \item Plus d'une centaine d'autorités dans les navigateurs
      \item Chacune peut certifier n'importe quel nom
      \item Le total est aussi fiable que la \emph{moins} fiable des racines !
    \end{itemize}
  \end{block}

  \begin{block}{Les autorités ne sont pas toujours fiables}
    \begin{itemize}
      \item En 2011, DigiNotar est compromis et émet des centaines de
        certificats frauduleux, qui seront utilisé pour MitM plus de 300000
        utilisateurs iraniens de Gmail.
      \item Racines retirées des navigateurs, compagnie en faillite
      \item Plusieurs incidents similaires au cours des années
      \item Problème : certains acteurs sont \emph{too big to fail}
      \item Problème : a-t-on détecté tous les incidents ?
    \end{itemize}
  \end{block}
\end{frame}

\begin{frame}{Solutions partielles émergentes (1)}
  \begin{block}{DANE (RFC 6698)}
    \begin{itemize}
      \item Distribution de certificates ou clés via le DNS
      \item S'ajoute aux CA traditionnelles ou les remplace
      \item Problème majeur : dépend de DNSSEC !
    \end{itemize}
  \end{block}

  \begin{block}{Certificate Transparency (RFC 6962)}
    \begin{itemize}
      \item Tous les certificats sont publiés dans un journal append-only
        vérifiable (arbre de Merkle)
      \item Les serveurs doivent fournit des preuves d'inclusion, ou bien les
        clients peuvent bavarder entre eux
      \item Tout le monde peut inspecter les journaux
      \item N'évite pas les problèmes, permet seulement de les détecter
    \end{itemize}
  \end{block}
\end{frame}

\begin{frame}{Solutions partielles émergentes (2)}
  \begin{block}{HTTP Public-Key Pinning (7469)}
    \begin{itemize}
      \item Le serveur peut imposer des clés qui devront être présentes dans
        la chaine lors des prochaines connections
      \item Complique l'administration lors du changement des clés
      \item Ne marche que pour HTTPS
      \item Ne sécurise pas la première connection
      \item En fait si pour certains sites et certains navigateurs (pinning
        pré-configurés)
    \end{itemize}
  \end{block}

  \begin{block}{Rapports de forces}
    Exemple récent : l'affaire Google vs Symantec
    \url{https://googleonlinesecurity.blogspot.fr/2015/10/sustaining-digital-certificate-security.html}
  \end{block}
\end{frame}

\section[SCA]{Attaques par canaux auxiliaires}
\tocsect

\begin{frame}[containsverbatim]{Le problème square-and-multiply}
  \begin{Verbatim}
    a = 1
    for i from 0 to bitlength(b) - 1 do:
        if b_i == 1 then:
            a = a * g
        a = a^2
  \end{Verbatim}
  \begin{itemize}
    \item RSA, (EC)DH et ECDSA comportent tous une étape cruciale qui
      s'implémente naïvement par square-and-multiply
    \item Le temps d'éxécution, à taille d'exposant donné, fuite directement
      le poids de Hamming de l'exposant secret
    \item Si on la corrige en square-and-multiply-always, si on ne fait pas
      gaffe on fuite la longueur de l'exposant secret (pour ECDSA c'est
      catastrophique)
    \item Des attaques plus sophistiquées peuvent observer les variations en
      fonction des valeurs de $g$
  \end{itemize}
\end{frame}

\begin{frame}{C'est exploitable en pratique}
  \begin{description}
    \item[1996] « Timing attacks on [...] RSA [...] » en local uniquement
    \item[2003] « Remote timing attacks are practical »
      \begin{itemize}
        \item Récupère la clé RSA complète du serveur
        \item Mis en pratique sur un réseau local
      \end{itemize}
    \item[2011] « Remote timing attacks are still practical »
      \begin{itemize}
        \item Récupère une clé ECDSA (sur certaines courbes dans OpenSSL)
        \item Mis en pratique sur un réseau local
      \end{itemize}
    \item[2009] « Opportunities and limits of remote timing attacks » : 100 ns
      sur un réseau local, 15-100 micro secondes sur l'internet
  \end{description}
\end{frame}

\begin{frame}{Autres attaques et défenses}
  \begin{block}{Autres canaux auxiliaires}
    \begin{itemize}
      \item Attaque par cache, en local ou à distance
      \item Avec un peu d'accès physique : \\
        \url{https://www.tau.ac.il/~tromer/acoustic/} \\
        \url{https://www.tau.ac.il/~tromer/handsoff/}
      \item Avec plus d'accès physique : puissance, ondes EM, etc.
    \end{itemize}
  \end{block}

  \begin{block}{Défense de base : codage en temps constant}
    \begin{itemize}
      \item Pas de branche qui dépend d'une donnée secrète
      \item Pas d'accès mémoire dont l'adresse dépend d'un secret
      \item Pas d'instruction à temps variable sur des secrets
      \item \url{https://cryptocoding.net/}
    \end{itemize}
  \end{block}

  Autre stratégie de défense : le \emph{blinding}
\end{frame}

\section[Échanges]{Autres échanges de clé}
\tocsect

\begin{frame}{Autres échanges de clé}
  \begin{block}{Clé pré-partagé : PSK}
    \begin{itemize}
      \item Sans crypto asymétrique, pour les environnement \emph{très}
        restreints
      \item Plus de travail pour la gestion des clés (il FAUT une clé par
        client)
      \item Pas de forward secrecy
    \end{itemize}
  \end{block}

  \begin{block}{Divers, peu utilisés}
    \begin{itemize}
      \item (EC)DH non-ephemère : léger gain de perf, perte de la FS
      \item DH_anon : léger gain de perf, vulnérable à un MitM
      \item DSS : aka DSA, nécessite certificat spécifique
      \item RSA_export, DH_export : limité à 512 bits, cf diapo suivante
    \end{itemize}
  \end{block}
\end{frame}

\begin{frame}{Attaques par downgrade sur les suites export}
  \begin{block}{FREAK (2015)}
    \begin{enumerate}
      \item L'attaquant modifie le ClientHello : RSA_export seulement
      \item Le serveur offre alors une clé RSA de 512 bits
      \item La clé RSA de 512 bits est cassée par l'attaquant (100\$)
      \item L'attaquant peut modifier les Finished pour effacer ses traces
      \item Solution évidente : désaciver RSA_export !
      \item C'est lent : \url{https://freakattack.com/}
    \end{enumerate}
  \end{block}

  \begin{block}{Logjam (2015)}
    \begin{itemize}
      \item Attaque similaire avec DH_export
      \item Un pré-calcul pour un $p$ donné permet de casser DH en direct
        pendant la poignée de mains
    \end{itemize}
  \end{block}
\end{frame}

\section[Attaques]{Autres attaques}
\tocsect

\begin{frame}{Attaques génériques}
  \begin{block}{Stripping}
    \begin{itemize}
      \item L'attaquant supprime une redirection HTTP \textrightarrow HTTPS
      \item L'attaquant supprime l'annonce de STARTTLS
      \item \url{http://www.thoughtcrime.org/software/sslstrip/}
      \item Solution partielle pour HTTP : HTTP-Strict-Transport-Security (RFC
        6797)
    \end{itemize}
  \end{block}

  \begin{block}{Dénis de service}
    \begin{itemize}
      \item Contre le serveur : l'attaquant envoie un ClientHello puis ne
        répond plus. Le serveur calcule une signature et un demi DH.
      \item Contre un tiers, en DTLS : un serveur mal configuré peut être
        utilisé comme amplificateur ; les cookies DTLS sont conçus pour
        l'empêcher.
    \end{itemize}
  \end{block}
\end{frame}

\begin{frame}{Renégociation}
  Consiste à effectuer une deuxième poignée de mains au sein d'une même
  connection (rafraichissement des clés, certificat client).

  \begin{block}{Renégociation non sécurisée (2009)}
    \begin{itemize}
      \item Les deux poignées de main ne sont pas liées cryptographiquement
      \item Un attaquant peut injecter du traffic avant le début d'une
        connection légitime
      \item Solution : RFC 5746
    \end{itemize}
  \end{block}

  \begin{block}{Triple handshake (2014)}
    \begin{itemize}
      \item Attaque sophistiquée permettant à un serveur de «
        voler » l'authentification du client auprès d'un autre serveur
      \item Trouvées en essayant de prouver TLS sûr\dots
      \item Solution : RFC 7627
    \end{itemize}
  \end{block}
\end{frame}

\begin{frame}{Bugs dans les implémentations}
  Sans doute plus de 80\% des failles de sécurité (au doigt mouillé) !
  Quelques exemples célèbres :
  \begin{itemize}
    \item La faille Debian/OpenSSL : de 2006 à 2008, mauvais patch, nombres
      aléatoires générés avec seulement 16 bits d'entropie
    \item Hearbleed (OpenSSL) : de 2012 à 2014, buffer overread, fuite
      d'information silencieuse, potentiellement clé secrète
      \url{http://heartbleed.com/}
    \item Early CCS (OpenSSL) : de 2012 à 2014, erreur de machine à état, MitM
      possible \url{https://www.imperialviolet.org/2014/06/05/earlyccs.html}
    \item Double goto fail d'Apple : (2014) \texttt{if(...) goto fail; goto
        fail; if(...)} ; permet d'utiliser des certificats sans leur clé
  \end{itemize}
  Liste non exhaustive !
\end{frame}

% \section{Conclusion}
% \tocsect
% 
% % rappel : primitives OK et cassées, tailles de clé
% % compromis : sécurité vs interop, perf/cout, admin, etc
% % aperçu TLS 1.3

\section{Pratique}
\tocsect

\begin{frame}{Pour la prochaine séance}
  \begin{itemize}
    \item Télécharger mbed TLS \url{https://tls.mbed.org/}
    \item Le compiler (cf Readme)
    \item Lire \url{https://tls.mbed.org/high-level-design}
    \item Lire \url{https://tls.mbed.org/kb/how-to/mbedtls-tutorial}
  \end{itemize}
\end{frame}

\end{document}
